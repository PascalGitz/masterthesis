\chapter*{Kurzfassung}

Die Masterthesis untersucht das nichtlineare Tragverhalten von Stahlbetontragwerken und entwickelt eine Methode zur präzisen Verformungsberechnung mit der Statiksoftware AxisVM. Der Fokus liegt dabei auf dem Biegetragverhalten von Balken- und Plattentragwerken.

Im ersten Hauptteil wird die Modellbildung detailliert beschrieben. Ein einführendes Beispiel illustriert die Grundlagen: Ein einfacher Balken wird mit starren Stäben modelliert, die durch Drehfedern verbunden sind. Den Drehfedern können nichtlineare Beziehungen zugewiesen werden, um das nichtlineare Tragverhalten abzubilden. Anschliessend wird ein Berechnungsschema vorgestellt, das die Modellbildung in zwei Schritte unterteilt: System- und Querschnittsmodellbildung. Die Systemmodellbildung umfasst die Anordnung der starren Stäbe, die Positionierung der Federn sowie sind die relevanten Resultatgrössen beschrieben. Die Querschnittsmodellbildung wird für Balken und Plattentragwerke separat beschrieben. Dabei werden in dieser die Bestimmung der nichtlinearen Biegesteifigkeit, die Transformation in eine Federbeziehung und ein Abbruchkriterium zur Definition des Versagens festgelegt.

Im zweiten Hauptteil wird die Anwendung der entwickelten Modelle anhand von vier Beispielen demonstriert: einem Zweifeldträger, einem torsionsweichen Trägerrost, einer Quadratplatte und der Nachrechnung eines Plattenexperiments. Die numerischen und analytischen Ergebnisse stimmen gut überein. Besonders die Nachrechnung des Plattenexperiments zeigt die zufriedenstellende Genauigkeit des Modells im Vergleich zu experimentellen Daten und einer Lösung mit der Software ANSYS.

Die Arbeit liefert eine fundierte Grundlage für die präzise Berechnung des nichtlinearen Tragverhaltens von Stahlbetontragwerken. Sie zeigt, wie existierende Softwarelösungen effizient eingesetzt werden können, um komplexe technische Fragestellungen zu lösen.