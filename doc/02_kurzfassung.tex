\chapter*{Kurzfassung}

Die vorliegende Masterthesis untersucht das nichtlineare Tragverhalten von Stahlbetontragwerken und entwickelt eine praxisorientierte Methode zur präzisen Verformungsberechnung. Ziel ist es, das Tragverhalten ausschliesslich mit der Statiksoftware AxisVM abzubilden. Diese Software wurde gewählt, da angenommen wird, dass die meisten in der Schweiz tätigen Ingenieure und Ingenieurinnen Zugang dazu haben. Der Fokus liegt auf dem Biegetragverhalten, welches für Balken- und Plattentragwerke analysiert wird.

Der erste Teil der Arbeit befasst sich mit der Modellbildung. Diese gliedert sich in ein allgemeines Kapitel, die Modellbildung von Balken und die Modellbildung von Platten. Im allgemeinen Kapitel illustriert ein einführendes Beispiel die Grundlagen, indem es das Verhalten eines einfachen Balkens mit starren Stäben und nichtlinearen Federbeziehungen darstellt. Das Berechnungsschema gliedert das Modell in die Systemmodellbildung und die Querschnittsmodellbildung. Zunächst ist die Systemmodellbildung aufgezeigt. Danach ist die Querschnittsmodellbildung für Balken mit der Ermittlung der  Biegesteifigkeit unter Berücksichtigung der Zugversteifung mit dem Zuggurtmodell beschrieben. Abschliessend zeigt die Querschnittsmodellbildung der Platten die Bestimmung der nichtlinearen Biege- und Drillsteifigkeit am Plattenelement und transformiert diese zum Trägerrost.

Der zweite Teil zeigt die Anwendung der Modelle an vier Beispielen: einem Zweifeldträger, einem torsionsweichen Trägerrost, einer Quadratplatte und der Nachrechnung eines Plattenexperiments. Die analytischen und numerischen Ergebnisse stimmen für die Berechnungsbeispiel überein. Die Versuchsnachrechnung zeigt eine gute Übereinstimmung mit den Messresultetn, sowie mit der Lösung mit ANSYS unter der Verwendung des CMM-Usermat. Dies bestätigt die Genaugikeit des Modells.

Die Thesis schliesst mit einer kritischen Würdigung. Sie zeigt, dass die Methode durch einfache Modellbildung und transparente Gelenkdefinitionen eine präzise und nachvollziehbare Analyse ermöglicht. Gleichzeitig wird der begrenzte Fokus auf das Biegetragverhalten angesprochen. Ein Ausblick hebt das Potenzial zur Materialeinsparung hervor, was angesichts der Nachhaltigkeitsdebatte von grosser Bedeutung ist.